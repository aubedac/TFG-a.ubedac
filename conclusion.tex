\chapter{Conclusiones}
\label{cap:conclusion}

\section{Competencias adquiridas}
Para el desarrollo de este Trabajo Fin de Grado he aplicado todos los conocimientos y competencias adquiridos durante la carrera a nivel de configuración de redes, análisis y diseño de escenarios simulados mediante el uso \textit{RadioMobile}, desarollo de código a bajo nivel y alto, utilizando en este último librerías para la extracción, procesado y representación de datos. Por consiguiente cómo objetivo de máximo interés, el haber aplicado todas estás tecnologías y competencias sobre un escenario real tanto cómo es el proyecto Napo y sobre los equipos NetMetal 5, realizando un estudio sobre el la red en conjunto y extraer conclusiones frente a la viabilidad y rendimiento de los equipos tratando de cumplir siempre con los objetivos del proyecto Napo.\\\\

Realizar este Trabajo Fin de Grado me ha aportado conocimientos generales en cuánto al diseño y despliegue de redes en países subdesarrollados. De igual forma, destacaría la capacidad de gestión y coordinación con los compañeros de la PUCP, los cuáles han formado parte activa durante desarrollo de este Trabajo Fin de Grado aportando datos y realizando pruebas de campo, para que la labor de diseño, análisis de pruebas y extracción de resultados sea lo más exacta posible. \\\\

Por último y como competencia clave, destacaría el poder trabajar con equipos reales a nivel de laboratorio, realizar pruebas a nivel de red, integrar los equipos con un \textit{software} de monitorización y en base a todo ello, extraer conclusiones y aportar valor desde un plano simulado en laboratorio sobre el rendimiento de los equipos en el escenario del proyecto Napo .

\section{Trabajos futuros}
En primer lugar y analizando el rendimiento ofrecido por lo equipos a nivel de laboratorio, sería interesante tratar de replicar el tráfico existente en las dos redes, tanto la de datos cómo la de telemedicina, realizando túneles MPLS para su diferenciación y agregación. De igual forma, se trataría de integrar la funcionalidad de esta red principal compuesta por los dipositivos NetMetal 5 con los nodos, cuya placa integrada es Alix, asociado a la creación de una red de \textit{backup}.\\\\

En segundo lugar, para conseguir una reproducción y extracción de valores más parejos frente al contexto del proyecto, deberían realizarse no sólo pruebas a nivel de laboratorio, si no que también dichas pruebas deberían ser realizadas con las distancias existentes en cada radioenlace del proyecto obteniendo así una reproducción más exacta en cuánto al rendimiento de los equipos.\\\\

Por último, realizar una integración con Zabbix del sistema de la red conjunto, teniendo en cuenta todos los dispositivos que están involucrados en la red y realizar una perfecta jerarquización de la misma. No sólo a la hora de integrar todo lo relacionado con alertas pudiendo crear asignaciones en función de las subredes existentes. De igual modo utilizar la aplicación que utiliza Zabbix respecto al \textit{discovery} para generar una base de datos representativa con todos los elementos que componen la red.