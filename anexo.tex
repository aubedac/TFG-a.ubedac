\chapter{Anexos}
\label{cap:anexo}
\section{Tablas empleadas}

\begin{table}[H]
	\begin{center}
		\begin{tabular}{|l|l|}
			\hline
			Enlaces Backhaul & Tráfico total (Kbps) \\
			\hline 
			Cabo Patonja - Torres Causana & 44528 \\ \hline
			Torres Causana – Tempestad & 44956 \\ \hline
			Tempestad – Tupac Amaru & 41572 \\ \hline
			Tupac Amaru – Angoteros & 39344 \\ \hline
			Angoteros – Campo Serio  & 37804 \\ \hline
			Campo Serio – Rumi Tuni & 35108 \\ \hline
			Rumi Tuni -  San Rafael & 33012 \\ \hline
			San Rafael – Copal Urco & 28604 \\ \hline
			Copal Urco – Santa Clotilde & 26112 \\ \hline
			Santa Clotilde – Tacsha Curaray & 24228 \\ \hline
			Tacsha Curaray – Negro Urco & 20044 \\ \hline
			Negro Urco – Tuta Pischo & 19316 \\ \hline
			Tuta Pishco – Huaman Urco & 12464 \\ \hline
			Huamán Urco – Mazán & 11260 \\ \hline
			Mazán – Iquitos & 5696 \\ \hline
			
		\end{tabular}
	\end{center}
	\caption{Tráfico total por enlace}
	\label{table:RLPUCP}
\end{table}

\begin{table}[H]
	\begin{center}
		\begin{tabular}{|l|l|l|}
			\hline
			Enlaces Backhaul & Latitud & Longitud\\
			\hline 
			Cabo Patonja & $^{\circ}$0 '58 "12,80 S & $^{\circ}$75 '10 "30,10 O  \\ \hline
			Torres Causana & $^{\circ}$1 '6 "15,40 S & $^{\circ}$75 '0 "15,80 O \\ \hline
			Tempestad & $^{\circ}$1 '16 "57,70 S & $^{\circ}$74 '53 "2,30 O \\ \hline
			Tupac Amaru & $^{\circ}$1 '21 "47,00 S & $^{\circ}$74 '44 "42,00 O \\ \hline
			Angoteros & $^{\circ}$1 '34 "6,80 S & $^{\circ}$74 '36 "40,60 O \\ \hline
			Campo Serio & $^{\circ}$1 '47 "40,80 S & $^{\circ}$74 '42 "28,60 O \\ \hline
			Rumi Tumi & $^{\circ}$2 '3 "14,00 S & $^{\circ}$74 '26 "10,50 O \\ \hline
			San Rafael & $^{\circ}$2 '21 "53,80 S & $^{\circ}$74 '6 "44,10 O \\ \hline
			Copal Urco & $^{\circ}$2 '20 "52,10 S & $^{\circ}$73 '47 "24,70 O \\ \hline
			Santa Clotilde & $^{\circ}$2 '29 "22,40 S & $^{\circ}$73 '40 "40,70 O \\ \hline
			Tacsha Curaray & $^{\circ}$2 '48 "47,60 S & $^{\circ}$73 '32 "27,20 O \\ \hline
			Negro Urco & $^{\circ}$3 '1 "23,10 S & $^{\circ}$73 '23 "31,50 O \\ \hline
			Tuta Pishco & $^{\circ}$3 '6 "31,40 S & $^{\circ}$73 '8 "17,50 O \\ \hline
			Huamán Urco & $^{\circ}$3 '19 "7,60 S & $^{\circ}$73 '13 "1,90 O \\ \hline
			Mazán & $^{\circ}$3 '29 "59,90 S & $^{\circ}$73 '5 "28,00 O \\ \hline
			Hospital Regional de Loreto & $^{\circ}$3 '36 "46,86 S & $^{\circ}$73 '10 "20,05 O \\ \hline
		\end{tabular}
	\end{center}
	\caption{Geolocación de los emplazamientos del NAPO}
	\label{table:geolocalizacion}
\end{table}

\begin{table}[H]	
	\begin{center}
		\begin{tabular}{|l|l|l|}
			\hline
			Enlace & Altura torres & Distancia \\
			\hline 
			Cabo Pantoja – Torres Causana & 45 m – 45 m & 24,11 Km \\ \hline	
			Torres Causana – Tempestad & 45 m – 60 m & 23,92 Km \\ \hline
			Tempestad – Tupac Amaru & 60 m – 45 m & 17,84 Km \\ \hline
			Tupac Amaru – Angoteros & 45 m – 66 m & 27,24 Km \\ \hline
			Angoteros – Campo Serio & 66 m – 66 m & 27,32 Km \\ \hline
			Campo Serio – Rumi Tumi & 66 m – 90 m & 41,72 Km \\ \hline
			Rumi Tumi – San Rafael & 90 m – 90 m & 49,89 Km \\ \hline
			San Rafael – Copal Urco & 90 m – 54 m & 35,82 Km \\ \hline
			Copal Urco – Santa Clotilde & 54 m – 72 m & 20,09 Km \\ \hline
			Santa Clotilde – Tacsha Curaray & 72 m – 72 m & 39,05 Km \\ \hline
			Tacsha Curaray – Negro Urco & 72 m – 75 m & 24,58 Km \\ \hline
			Negro Urco – Tuta Pishco & 75 m- 57 m & 29,75 Km \\ \hline
			Tuta Pishco – Huamán Urco & 57 m – 66 m & 24,93 Km \\ \hline
			Huamán Urco – Mazán & 66 m – 69 m & 24,52 Km \\ \hline
			Mazán - Hospital Regional Loreto & 69 m - 30 m & 15,45 Km \\ \hline
		\end{tabular}
	\end{center}
	\caption{Altura de torres y distancia entre emplazamientos del NAPO}
	\label{table:distancias}
\end{table}

\begin{table}[H]
	\begin{center}
		\begin{tabular}{|l|l|l|}
			\hline
			Enlace & Mbps obtenida & Distancia \\
			\hline 
			Cabo Pantoja – Torres Causana & 53,34973333333333 & 24,11 Km \\ \hline	
			Torres Causana – Tempestad & 53,367466666666665 & 23,92 Km \\ \hline
			Tempestad – Tupac Amaru & 53,93493333333333 & 17,84 Km \\ \hline
			Tupac Amaru – Angoteros & 53,0576 & 27,24 Km \\ \hline
			Angoteros – Campo Serio & 53,05013333333333 & 27,32 Km \\ \hline
			Campo Serio – Rumi Tumi & 51,70613333333333 & 41,72 Km \\ \hline
			Rumi Tumi – San Rafael & 50,943599999999996 & 49,89 Km \\ \hline
			San Rafael – Copal Urco & 52,2568 & 35,82 Km \\ \hline
			Copal Urco – Santa Clotilde & 53,72493333333333 & 20,09 Km \\ \hline
			Santa Clotilde – Tacsha Curaray & 51,95533333333333 & 39,05 Km \\ \hline
			Tacsha Curaray – Negro Urco & 53,30586666666667 & 24,58 Km \\ \hline
			Negro Urco – Tuta Pishco & 52,82333333333333 & 29,75 Km \\ \hline
			Tuta Pishco – Huamán Urco & 53,273199999999996 & 24,93 Km \\ \hline
			Huamán Urco – Mazán & 53,31146666666667 & 24,52 Km \\ \hline
			Mazán - Hospital Regional Loreto & 54,158 & 15,45 Km \\ \hline
		\end{tabular}
	\end{center}
	\caption{Valores de capacidad teórica respecto a cada radioenlace utilizando una MCS10 para NV2 y 40 Mhz}
	\label{table:capacidades}
\end{table}

\begin{table}[H]
	\begin{center}
		\begin{tabular}{|l|l|l|l|}
			\hline
			Enlace & Tráfico Tx (Mbps) & Tráfico Rx (Mbps) & Tráfico total (Mbps) \\
			\hline 
			Santa Clotilde – Tacsha Curaray & 49,5 & 33,7 & 83,2  \\ \hline
			Tacsha Curaray – Negro Urco & 55,8 & 61,3 & 117,1 \\ \hline
			Negro Urco – Tuta Pishco & 13,3 & 4,7 & 18 \\ \hline
			Tuta Pishco – Huamán Urco & 29,7 & 43,8 & 73,5 \\ \hline
			Huamán Urco – Mazán & 33,7 & 28,8 & 62,5 \\ \hline
			Mazán - Iquitos & 41,3 & 50,8 & 92,1 \\ \hline
		\end{tabular}
	\end{center}
	\caption{Valores instántaneos obtenidos en campo }
	\label{table:medidasInstNapo}
\end{table}

\begin{table}[H]
	\begin{center}
		\begin{tabular}{|l|l|l|l|}
			\hline
			Enlace & Tráfico Tx (Mbps) & Tráfico Rx (Mbps) & Tráfico total (Mbps)\\
			\hline 
			Santa Clotilde – Tacsha Curaray & 67,6 & 56,3 & 123,9\\ \hline
			Tacsha Curaray – Negro Urco & 48,6 & 51,5 & 100,1 \\ \hline
			Negro Urco – Tuta Pishco & 6,4 & 3,7 & 10,1 \\ \hline
			Tuta Pishco – Huamán Urco & 32,5 & 36 & 68,5 \\ \hline
			Huamán Urco – Mazán & 30,3 & 25,6 & 55,9 \\ \hline
			Mazán - Iquitos & 30,6 & 43,5 & 74,1 \\ \hline
		\end{tabular}
	\end{center}
	\caption{Valores medios obtenidos en campo}
	\label{table:medidasMediasNapo}
\end{table}

\begin{table}[H]
	\begin{center}
		\begin{tabular}{|l|l|l|l|}
			\hline
			Enlace & Tráfico Tx (Mbps) & Tráfico Rx (Mbps) & Tráfico total (Mbps)\\
			\hline 
			Santa Clotilde – Tacsha Curaray & 71,2 & 66,2 & 137,4\\ \hline
			Tacsha Curaray – Negro Urco & 71,2 & 66,1 & 137,3 \\ \hline
			Negro Urco – Tuta Pishco & 71,3 & 66,3 & 137,6 \\ \hline
			Tuta Pishco – Huamán Urco & 71,6 & 67 & 138,6 \\ \hline
			Huamán Urco – Mazán & 69,6 & 66,2 & 135,8 \\ \hline
			Mazán - Iquitos & 71,2 & 67,1 & 138,3 \\ \hline
		\end{tabular}
	\end{center}
	\caption{Valores instántaneos obtenidos en Laboratorio}
	\label{table:medidasInstLabNapo}
\end{table}

\begin{table}[H]
	\begin{center}
		\begin{tabular}{|l|l|l|l|}
			\hline
			Enlace & Tráfico Tx (Mbps) & Tráfico Rx (Mbps) & Tráfico total (Mbps)\\
			\hline 
			Santa Clotilde – Tacsha Curaray & 57,4 & 47,2 & 104,6 \\ \hline
			Tacsha Curaray – Negro Urco & 55,5 & 46,5 & 102 \\ \hline
			Negro Urco – Tuta Pishco & 58 & 43,9 & 101,9 \\ \hline
			Tuta Pishco – Huamán Urco & 56,9 & 50,3 & 107,2 \\ \hline
			Huamán Urco – Mazán & 60,1 & 55,8 & 115,9 \\ \hline
			Mazán - Iquitos & 43,5 & 43,1 & 86,6 \\ \hline
		\end{tabular}
	\end{center}
	\caption{Valores medios obtenidos en laboratorio}
	\label{table:medidasMediasLabNapo}
\end{table}

\begin{table}[H]
	\begin{center}
		\begin{tabular}{|l|l|}
			\hline
			Modulación & Sensibilidad (dBm)\\
			\hline 
			MCS8 & -95 \\ \hline
			MCS9 & -93 \\ \hline
			MCS10 & -90 \\ \hline
			MCS11 & -87 \\ \hline
			MCS12 & -84 \\ \hline
			MCS13 & -79 \\ \hline
			MCS14 & -78 \\ \hline
			MCS15 & -75 \\ \hline
		\end{tabular}
	\end{center}
	\caption{Valores teóricos de sensibilidad para cada MCS utilizando el protocolo NV2}
	\label{table:sensibilidadMCS}
\end{table}


\begin{table}[H]
	\begin{center}
		\begin{tabular}{|l|l|l|}
			\hline
			Enlace & Potencia teórica (dBm) & Potencia práctica (dBm)\\
			\hline 
		Santa Clotilde – Tacsha Curaray & -78,1 & -67 \\ \hline
		Tacsha Curaray – Negro Urco & -54,5 & -45 \\ \hline
		Negro Urco – Tuta Pishco & -57,4 & -75  \\ \hline
		Tuta Pishco – Huamán Urco & -54,3 & -68 \\ \hline
		Huamán Urco – Mazán & -70,6 & -68 \\ \hline
		Mazán - Iquitos & -49,5 & -66 \\ \hline
		\end{tabular}
	\end{center}
	\caption{Potencia de señal obtenida según radioenlace de forma teórica y práctica}
	\label{table:sensibilidadCampoObtenida}
\end{table}

\begin{table}[H]
	\begin{center}
		\begin{tabular}{|l|l|l|}
			\hline
			 & NetMetal 5 & RM-MAX\\
			\hline 
		Máx. Consumo de potencia  & 23 W & 8 W \\ \hline
		Potencia de entrada & 8 - 30 VDC & 48 VDC \\ \hline
		Frecuencia de operación & 4920 - 6100 MHzs & 2,4 - 2,5 GHzs | 5125 - 5875 MHzs  \\ \hline
		MIMO & 3x3 & 2x2 \\ \hline
		Máx. Potencia de transmisión & 33 dBm & 23 dBm \\ \hline
		Ancho de canal & 20/40/80 MHzs & 5/10/20/40 MHzs \\ \hline
		Máx. \textit{Throughput} & 1300 Mbps & 50 Mbps \\ \hline
		\end{tabular}
	\end{center}
	\caption{Parámetros de los equipos NetMetal 5 y RM-MAX}
	\label{table:parámetrosEquipos}
\end{table}

\begin{table}[H]
	\begin{center}
		\begin{tabular}{|l|l|l|}
			\hline
			 & PowerBeam AC & AirFiber\\
			\hline 
		Máx. Consumo de potencia  & 8.5 W & 12 W \\ \hline
	    Rango de Frecuencias (MHzs)| PIRE Máx (dBm)  & 5150 - 5875 | 53 dBm & \makecell{5150 - 5250 | 49 dBm \\ 5250 - 5350 | 30 dBm \\ 5470 - 5725 | 30 dBm \\ 5725 - 5850 |36, 54 dBm } \\ \hline
	    
		Máx. Potencia de transmisión & 24 dBm & 26 dBm \\ \hline
		\end{tabular}
	\end{center}
	\caption{Parámetros de los equipos PowerBeam AC y AirFiber}
	\label{table:parámetrosEquiposUbiquiti}
\end{table}

\begin{table}[H]
	\begin{center}
		\begin{tabular}{|l|l|l|}
			\hline
			 & NetMetal 5 & RM-MAX\\
			\hline
            Ancho de canal & 20 - 40 MHzs & 20 - 40 MHzs\\
			\hline 
		Promedio pérdida paquete  & 12,68\% - 0,46\% & 46\% - 63,66\% \\ \hline
		Promedio retardo (ms) & 11,47 - 10,69 & 223,52 - 168,27 \\ \hline
		Promedio Bandwidth TCP (Mbps) & 10,12 - 19,30  & 1,44 - 0.005  \\ \hline
		Promedio Bandwidth UDP (Mbps) & 14,25 - 24,32  & 0,03 - 0.006  \\ \hline
		\end{tabular}
	\end{center}
	\caption{Resultados obtenidos al realizar pruebas sobre los equipos NetMetal 5 y RM-MAX}
	\label{table:pruebasEquipos}
\end{table}

\begin{table}[H]
	\begin{center}
		\begin{tabular}{|l|l|l|}
			\hline
			 & PowerBeam AC  & AirFiber 5x\\
			\hline
            Ancho de canal & 20 MHzs & 40 - 50 MHzs\\
			\hline 
		Promedio pérdida paquete  & 40\% & 20\% \\ \hline
		Promedio retardo (ms) & 183.07  & 10.31 \\ \hline
		Promedio Bandwidth TCP (Mbps) & 40  & 37 - 78 \\  \hline
		\end{tabular}
	\end{center}
	\caption{Resultados obtenidos al realizar pruebas sobre los equipos Ubiquiti}
	\label{table:pruebasEquiposUbiquiti}
\end{table}

\section{Código utilizado}
El siguiente código es el utilizado para la representación de \textit{throughput} en función de la distancia  y MCS, guardado en el archivo \texttt{capacidades.py}:
\begin{lstlisting}[caption={Script utilizado para representar \textit{throughput} en función de la distancia.}\label{lst:script},captionpos=b][language=Python]
import csv
import pandas as pd
import math 
import numpy as np
import matplotlib.pyplot as plt
import matplotlib.cm as cm

dataset = pd.read_csv('localidades.csv', delimiter = ',')
dataset = pd.DataFrame(dataset)
dataset1 = pd.read_csv('medidas NV2 20Mhz.csv', delimiter=',')
dataset1 = pd.DataFrame(dataset1)
colors = cm.rainbow(np.linspace(0, 1, 15))
nombres_rl = dataset['Localidad (A-B)']
nombres_mcs = ['MCS8', 'MCS9', 'MCS10', 'MCS11', 'MCS12']
distances = dataset['Distancia (Km)']

distances_array = []
for d in distances:
d = float(d.replace(",","."))
distances_array.append(np.float(d))

coor_x1 = []
coor_x2 = []
coor_y1 = []
coor_y2 = []

dataset1 = dataset1[:5]
for x, y in zip(dataset1['0'], dataset1['30']):
coor_x1.append(0.0)
coor_y1.append(x.replace(",","."))
coor_x2.append(30.0)
coor_y2.append(y.replace(",","."))

coordenadas = [coor_x1, coor_y1, coor_x2, coor_y2]
coordenadas = np.transpose(coordenadas)

throughput_mods_20 = []

def throughput(fila,d):
return  ((np.float(d) - np.float(fila[0])) * (( (np.float(fila[3]))) - (( np.float(fila[1])))) / (np.float(fila[2])-np.float(fila[0])) + np.float(fila[1])*2)

for i in range(0, len(coordenadas)):
fila = coordenadas[i]
for d in range(0, len(distances_array)):
throughput_mods_20.append((throughput(fila, distances_array[d])))

throughput_rl_20 = np.reshape(throughput_mods_20, (5,15)).T

def printable(array):
print_var = []
for item in array:
print_var += [item]
return print_var

rl_mcs8 = throughput_rl_20[:,0]
rl_mcs9 = throughput_rl_20[:, 1]
rl_mcs10 = throughput_rl_20[:, 2]
rl_mcs11 = throughput_rl_20[:, 3]
rl_mcs12 = throughput_rl_20[:,4]

printable_mcs8 = printable(rl_mcs8)
printable_mcs9 = printable(rl_mcs9)
printable_mcs10 = printable(rl_mcs10)
printable_mcs11 = printable(rl_mcs11)
printable_mcs12 = printable(rl_mcs12)

fig, ax = plt.subplots(figsize=(22,9))
plt.xlabel('Distance(KM)', fontsize=8)
plt.ylabel('C (Mbps)', fontsize=8)

plt.plot(distances_array, printable_mcs8, '-', label=nombres_mcs[0])
plt.plot(distances_array, printable_mcs9, '-',  label=nombres_mcs[1])
plt.plot(distances_array, printable_mcs10, '-', label=nombres_mcs[2])
plt.plot(distances_array, printable_mcs11, '-', label=nombres_mcs[3])
plt.plot(distances_array, printable_mcs12, '-', label=nombres_mcs[4])

ax.legend(ncol=2)    
ax.grid(True)
plt.show()
fig.savefig('rectas.png', format='png', dpi=1000)

fig_1, ax = plt.subplots(figsize=(22,9))
plt.xlabel('Distance(KM)', fontsize=8)
plt.ylabel('C (Mbps)', fontsize=8)

for i, c in zip(range(0, 14), colors):
ax.scatter(distances_array[i], printable_mcs8[i], color=c, label=nombres_rl[i])
ax.scatter(distances_array[i], printable_mcs9[i], color=c)
ax.scatter(distances_array[i], printable_mcs10[i], color=c)
ax.scatter(distances_array[i], printable_mcs11[i], color=c)
ax.scatter(distances_array[i], printable_mcs12[i], color=c)

ax.legend(ncol=4)    
ax.grid(True)
plt.show()
fig_1.savefig('valoresmbps.png',format='png', dpi=1000)
\end{lstlisting}