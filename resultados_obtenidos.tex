\chapter{Resultados obtenidos}
\label{cap:resultaldos_obtenidos}

En este capítulo se expondrán los datos y conclusiones obtenidos durante la realización de este Trabajo de Fin de Grado. Para llevar a cabo el desarrollo de este capítulo, no sólo se ha tenido en cuenta los resultados y pruebas realizados en entornos simulados, sino que también los aportados por la PUCP. Dichos datos aportados por la PUCP se componen de resultados y configuraciones realizadas en el entorno real del río Napo.\\\\

Por un lado se expondrán los datos obtenidos en el entorno simulado, es decir utilizaremos la configuración y herramientas mencionadas en el capítulo \ref{cap:metodologia} para realizar las pruebas respecto a la tasa de transmisión y recepción de los equipos. De foma complementaria se utilizará la herramienta \textit{RadioMobile} para obtener una representación de la red obteniendo una estimación del balance de enlace de la red.\\\\ 

Por otro lado se recogerán los valores proporcionados por la PUCP durante la realización de pruebas y se contrastarán con las medidas obtenidas en laboratorio. Cabe destacar que las pruebas mencionadas sólo han podido realizarse en un conjunto de radioenlaces determinados. Dichos enlaces son los comprendidos entre Santa Clotilde e Iquitos.\\\\

Por último y como objetivo final de este capítulo, se tratará de explicar ambos conjuntos de datos para así poder realizar un análisis del estado de cada radioenlace perteneciente a la red del Napo, teniendo en cuenta el objetivo explicado en el capítulo \ref{cap:introduccion}.
 
\section{Análisis de resultados}
Para llevar a cabo el desarrollo y análisis de los resultados hemos tomado como configuración referencia los valores proporcionados por la PUCP, los cúales se detallan a continuación:
\begin{itemize}
	\item Tiempo de la observación: 25 segundos.
	\item Modulación empleada: MCS10
	\item Frecuencia: 5805 Hz
	\item Estándar: 802.11n
	\item Ancho de banda: 40 MHzs
\end{itemize}
De igual forma se ha utilizado la herramienta propuesta por la PUCP, que en este caso es la propietaria de MikroTik descrita en el capítulo \ref{cap:metodologia}. Para llevar a cabo las pruebas respecto a la capacidad de cada radioenlace  utilizaremos el siguiente comando en la terminal de los equipos, que está compuesto por los siguientes parámetros: dirección destino, sentido de la comunicación y tiempo de observación:
\begin{lstlisting}
 tool bandwidth-test xx.xx.xx.xx direction=both duration=25
\end{lstlisting}

Una vez concretado los parámetros y configuraciones base de las pruebas que vamos a realizar, es necesario conocer los valores que se han obtenido en las pruebas de campo y contrastarlos con los valores teóricos que hemos obtenido con la simulación. En este caso aparte de los valores respecto a la capacidad del enlace, es importante conocer la potencia en recepción que está recibiendo cada radioenlace y analizar si dichos valores carecen o no de sentido. Por tanto, en primer lugar haremos un primer análisis tomando como referencia los datos obtenidos en simulación y en campo respecto a la potencia de señal recibida y en segunda lugar realizaremos un análisis similar, salvo que en este caso el objeto de análisis será la capacidad del enlace.\\\\

Para poder realizar una comparación coherente entre los datos aportados utilizaremos, por un lado los datos teóricos respecto a la sensibilidad mínima necesaria para cada modulación utilizando el protocolo NV2, tal y como se muestra en la tabla \ref{table:sensibilidadMCS}. Y por otro lado, los datos obtenidos de la simulación hecha con \textit{RadioMobile} y los valores proporcionados por los equipos durante la realización de pruebas, que se muestran en la tabla \ref{table:sensibilidadCampoObtenida}.\\
En este caso nos interesa centrarnos en el valor correspondiente a la modulación MCS10 ya que es la cual ha sido utilizada para realizar pruebas. Dicha modulación tiene una sensibilidad mínima de -90 dBm, por tanto para asegurar un correcto rendimiento del enlace utilizando esa modulación cada radioenlace debe estar por encima de dicho valor. Cómo vemos, para los valores teóricos y empíricos de los radioenlaces comprendidos entre Cabo Pantoja e Iquitos se cumple, por tanto podemos utilizar la modulación MCS10 para transmitir.\\\\

De igual forma los valores simulados han sido obtenidos teniendo en cuenta los parámetros tiempo y situaciones que ofrece \textit{RadioMobile},los cuáles son editables y en este caso han sido fijados con valores de 99\% y 80\%, respectivamente. La edición de estos parámetros implica una variación en los niveles de señal medio afectando de manera directa en las condiciones estadísticas que definen al modelo. En resumen, esto se traduce a que durante el 80\% del tiempo la atenuación no excederá el valor que obtengamos, al menos, un 99\% del tiempo.\\\\

En conclusión, observamos como para una distancia cercana a los 30 Km existe una medida anómala, esta medidas es la obtenida en el enlace de Negro Urco - Tuta pishco ya que los datos experimentales obtenidos distan mucho de los teóricos. Aunque se sobrepase el límite de -90 dBm que marca la modulación MCS10 se debería corregir o hacer un reajuste a ese radioenlace puesto que puede poner en compromiso el rendimiento total de la red.\\\\

A continuación para completar el análisis realizaremos un procedimiento similar al anterior pero teniendo en cuenta los valores obtenidos en las pruebas realizadas frente a la capacidad de cada enlace. En este análisis serán utilizados los valores obtenidos en el entorno de laboratorio y los proporcionados por la PUCP, los cúales se recogen en las tablas \ref{table:medidasInstLabNapo}, \ref{table:medidasMediasLabNapo}, \ref{table:medidasInstNapo} y \ref{table:medidasMediasNapo} respectivamente. \\\\

En resumen, y analizando el conjunto de las pruebas realizadas y aportadas por la PUCP se destaca lo siguiente:

\begin{itemize}
	\item El enlace de Santa Clotilde - Tacsha Curaray no sólo obtiene mejores tasas pese a ser uno de los más largo, sino que también obtiene una mayor potencia de señal recibida que los enlaces que son más cortos. Esto pone en duda su configuración puesto que existen enlaces cuya distancia es aproximadamente la mitad y consiguen unas tasas bastante inferiores. En este caso, debería revisarse la configuración de los equipos y las instalaciones para conocer el motivo de dichos datos. 
	
	\item El enlace Tacsha Curaray - Negro Urco también es llamativo, aunque en este caso por los motivos opuestos a lo comentado en el enlace de Santa Clotilde. Siendo un enlace cuya distancia es intermedia, en torno a 25 Km, obtiene unos valores de señal recibida y tasa muy altos en comparación. La explicación a esto podría deberse a las reflexiones de los elementos geológicos existentes entre el enlace, aunque es llamativa esa diferencia tan remarcada de potencia de señal recibida entre los enlaces de una distancia similar.
	
	\item El enlace Negro Urco - Tuta Pishco siendo un enlace cercano a los 30 km, es decir no siendo un enlace excesivamente largo, obtiene una tasa muy inferior a la esperada aunque su señal de potencia recibida sea la más baja de todos los enlaces analizados. En similares condiciones, enlaces con las mismas características obtienen un rendimiento mucho mejor que este, de tal forma que existe un problema en este enlace, bien sea de configuración de equipos o de apuntamiento de antenas.
	
	\item Respecto a los valores obtenidos en laboratorio y los posibles alcanzados en el escenario real del proyecto, el rendimiento de los equipos \textit{MikroTik} podría asegurar los objetivos definidos en el capítulo \ref{cap:introduccion} de poder mantener la cohesión de las dos redes. 
\end{itemize}
