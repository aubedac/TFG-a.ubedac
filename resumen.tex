\chapter*{Resumen}
\thispagestyle{empty}
\label{cap:resumen}

Hoy en día, las sociedades a nivel global dependen cada vez más de las Tecnologías de la Información y las Comunicaciones (TIC) para una cantidad cada vez mayor de actividades y servicios, hasta el punto de que, a nivel global, en muchas sociedades la falta de acceso a las redes de comunicaciones puede ya considerarse una razón de exclusión social severa. No obstante, en muchos países en vías de desarrollo hay una fuerte brecha entre los recursos TIC accesibles en zonas urbanas y rurales, ya que los operadores de telecomunicación no invierten en zonas rurales alejadas de los focos de concentración de la población debido a la fuerte inversión que se requiere para obtener muy poco retorno.\\

Desde 2007, el esfuerzo conjunto de varias entidades, entre las que se encontró la URJC, posibilitó que se desplegara una red de telemedicina a lo largo del río Napo, en la Amazonía Peruana, desde Cabo Pantoja hasta Iquitos, lugar donde se encuentra el Hospital Regional de Loreto. Dicha red posibilitaba la telemedicina en los puestos rurales de salud a lo largo de 450 Km de rivera del río. Más adelante, entre 2012 y 2016, esta red fue repotenciada y, en una pequeña parte de ella (tres poblaciones), el proyecto TUCAN3G habilitó el acceso a 3G con Telefónica del Perú para la población en general. El siguiente objetivo que se estableció fue la generalización del servicio 3G al resto de poblaciones, manteniendo además una convivencia óptima y sostenible entre la red de telemedicina y la red de backhaul del operador.\\

Este proyecto se basa en el diseño y estudio de la extensión e integración de las dos redes anteriormente mencionadas. Para ello se tendrá en cuenta todo lo desarrollado en el proyecto TUCAN3G así como las nuevas aportaciones existentes desde la Pontificia Universidad Católica del Perú (PUCP), con el fin de realizar un despliegue que satisfaga los objetivos marcados en este proyecto.\\

Para llevar a cabo todo el proyecto, se recreará en laboratorio un escenario similar al existente en la cuenca del río Napo, con el objetivo de estudiar el rendimiento y viabilidad de utilizar los nuevos equipos propuestos. Además de esto, se analizarán diferentes soluciones \textit{software} para la gestión y monitorización de la nueva red del proyecto.\\

\nocite{simo2015sharing}
\nocite{bebea2010diseno}
\nocite{martinez2016tucan3g}
\nocite{rey2011telemedicine}
\nocite{botta2013d}
\nocite{salguero2009propuesta}
\nocite{tucan3gd52}
\nocite{tucan3gd51}
\nocite{molisch2012wireless}
\nocite{nstremeFigures}
\nocite{ijcat}
\nocite{gsyc}
\nocite{eccs}
\nocite{miro}

\afterpage{\null\newpage}